\documentclass{beamer}

% Theme choice:
\usetheme{Madrid}
\usepackage{setspace}
\usepackage{amssymb}
\usepackage{amsmath}
\usepackage{amsthm}
\usepackage{enumitem}
\usepackage{mathrsfs}
\usepackage{mathtools}
\usepackage{enumerate}
\usepackage{calc}                     

\providecommand{\abs}[1]{\left\vert#1\right\vert}
\providecommand{\pr}[1]{\ensuremath{\Pr\left(#1\right)}}
\providecommand{\cbrak}[1]{\ensuremath{\left\{#1\right\}}}
\providecommand{\brak}[1]{\ensuremath{\left(#1\right)}}
\providecommand{\sbrak}[1]{\ensuremath{{}\left[#1\right]}}

\title{Assignment 13} 
\author{Varshini Jonnala - CS21BTECH11024}
\date{June 15, 2022}

\begin{document}
\begin{frame}
    \titlepage 
\end{frame}

\section{Question}
\begin{frame}{Question}

    Let $P_n(z)$ represent the Levinson polynomial of the first kind. 
    \begin{enumerate}[label=(\alph*)]
        \item If one of the roots of $P_n(z)$ lie on the unit circle, then show that all other roots of $P_n(z)$ are simple and lie on the unit circle.
        \item If the reflection coefficient $s_k\neq0$, then show that $P_k(z)$ and $P_{k+1}(z)$ have no roots in common. 
    \end{enumerate}
\end{frame}

\section{Solution}
\begin{frame}{Solution}

    (a) Let $z=e^{j\theta_1}$ represent one of the roots of the Levinson Polynomial $P_n(z)$ that lie on the unit circle. In that case,
    \begin{align}
        P_n\brak{e^{j\theta_1}} &= 0
    \end{align}
    
    and substituting this into the recursion equation, 
    \begin{align}
        \sqrt{1-s_n^2}P_n(z) &= P_{n-1}(z)-z s_n\tilde{P}_{n-1}(z)
    \end{align}
    
    We get
    \begin{align}
        \abs{s_n} &= \abs{\frac{P_{n-1}\brak{e^{j\theta_1}}}{\tilde{P}_{n-1}\brak{e^{j\theta_1}}}}=1
    \end{align}
    
    so that
    \begin{align}
        s_n &= e^{j\alpha}
    \end{align}
\end{frame} 

\begin{frame}
    Let 
    \begin{align}
        P_{n-1}(e^{j\theta}) &= R(\theta)e^{j\psi(\theta)}
    \end{align}

    and since $P_{n-1}(z)$ is free of zeros in $|z|\leq 1$,we have $R(\theta)>0,0<\theta<2\pi$, and once again substituting these into 
    \begin{align}
        \sqrt{1-s_n^2}P_n(z) &= P_{n-1}(z)-z s_n\tilde{P}_{n-1}(z)
    \end{align}
    
    We obtain
    \begin{align}
        \sqrt{1-s_n^2}P_n(e^{j\theta}) &= R(\theta)e^{j\psi(\theta)} - e^{j(\theta+\alpha)}e^{j(n-1)\theta}R(\theta)e^{-j\psi(\theta)}\\
        &= R(\theta)[e^{j\psi(\theta)}-e^{j(n\theta+\alpha)}e^{-j\psi(\theta)}]\\
        &= 2jR(\theta)e^{j(n\theta+\alpha)/2} sin\brak{\psi(\theta)-\frac{n\theta}{2}-\frac{\alpha}{2}}
    \end{align}
\end{frame}


\begin{frame}
    We observe that,
    \begin{enumerate}[label=(\roman*)]
        \item Due to the strict Hurwitz nature of $P_{n-1}(z)$, as $\theta$ varies from $0$ to $2\pi$,there is no net increment in the phase term $\psi(\theta)$, and the entire argument of the sine term above increases by $n\pi$.
        \item Consequently $P_n(e^{j\theta})$ equals zero atleast at $n$ distinct points $\theta_1,\theta_2,\dots,\theta_n, 0<\theta_i<2\pi$.
        \item However $P_n(z)$ is a polynomial odd degree $n$ is $z$ and can have atmost $n$ zeros. Thus all the above zeros are simple and they all lie on the unit circle. 
    \end{enumerate}
\end{frame}

\section{Solution - (b)}
    \begin{frame}{Solution - (b)}
    (b)  Suppose $P_n(z)$ and $P_{n-1}(z)$ has a common zero at $z=z_0$. Then $\abs{z_0}>1$ and we obtain
    \begin{align}
        z_0s_n\tilde{P}_{n-1}(z_0) &=0
    \end{align}
    which gives $s_n=0$, since $\tilde{P}_{n-1}(z_0)=0$ as it has all its zeros in $|z|<1$.\\
    Hence $s_n \neq 0$ implies $P_n(z)$ and $P_{n-1}(z)$ do not have any common roots.
\end{frame}

\end{document}
