\documentclass[12pt]{article}
\usepackage[utf8]{inputenc}
\usepackage{amssymb}
\usepackage{amsmath}
\usepackage{graphicx}

\title{AI1110 ASSIGNMENT-1}
\author{CS21BTECH11024 - Jonnala Varshini}

\begin{document}
  \maketitle
  \section*{ICSE 10 2018}
  \section*{Question 7(c)}
  
  \includegraphics[width=\textwidth]{prv1a.png}
  \section*{Solution:}
 According to the question, $M$ is a point on the side $AB$ such that $$AM : MB = 1 : 2$$
  It is known that, \\
  When the line segment AB is divided internally by C in the ratio m:n,\\
  we use Section formula to find the point C.\\\\
  The Coordinates of point C will be,
  [$\frac{mx2+nx1}{m+n}$, $\frac{my2+ny1}{m+n}$], where $A(x1,y1)$,$B(x2,y2)$ \\\\
  From given data, using Section formula, we get\\\\
    $M$ = ($\frac{-1+4}{1+2}$,$\frac{2+10}{1+2}$) = $(1,4)$\newpage
  The equation of the line joining two points (a,b),(c,d) is \\
        (y-b) = {$\frac{d-b}{c-a}$}(x-a)\\\\
  Here, the equation of the line joining $C(5,8)$ and $M(1,4)$ will be\\
      (y-4) = $\frac{8-4}{5-1}$(x-1) \\\\
  Simplified, we get the equation $$x-y+3=0$$
 \subsection*{But, However,} On calculating, we get\\
     The equation of the line joining $A(2,5), B(-1,2)$ as $x-y+3=0$ and \\
     the equation of the line joining $B(-1,2), C(5,8)$ as $x-y+3=0$ too.\\
     This implies that A,B,C are `collinear' and pass through $x-y+3=0$ and hence, given points $A,B,C$ don't form a triangle.\newline
    Verifying by plotting the graph of A,B,C and M points :
  \begin{center}
       \includegraphics[width=\textwidth]{prv1b.png}
  \end{center}
    
\end{document}
