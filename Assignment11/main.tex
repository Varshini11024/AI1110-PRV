\documentclass{beamer}

% Theme choice:
\usetheme{Madrid}
\usepackage{setspace}
\usepackage{amssymb}
\usepackage{amsmath}
\usepackage{amsthm}
\usepackage{enumitem}
\usepackage{mathrsfs}
\usepackage{mathtools}
\usepackage{longtable}
\usepackage{graphicx}

\usepackage{listings}
    \usepackage{color}                                            %%
    \usepackage{array}                                            %%
    \usepackage{longtable}                                        %%
    \usepackage{calc}                                             %%
    \usepackage{multirow}                                         %%
    \usepackage{hhline}                                           %%
    \usepackage{ifthen}                                           %%
    \usepackage{lscape}     
    \usepackage{amsmath}
\def\inputGnumericTable{}
\newcommand{\mydet}[1]{\ensuremath{\begin{vmatrix}#1\end{vmatrix}}}
\newcommand{\myvec}[1]{\ensuremath{\begin{pmatrix}#1\end{pmatrix}}}
\providecommand{\pr}[1]{\ensuremath{\Pr\left(#1\right)}}
\providecommand{\cbrak}[1]{\ensuremath{\left\{#1\right\}}}
\providecommand{\brak}[1]{\ensuremath{\left(#1\right)}}
\newcommand*{\permcomb}[4][0mu]{{{}^{#3}\mkern#1#2_{#4}}}
\newcommand*{\perm}[1][-3mu]{\permcomb[#1]{P}}
\newcommand*{\comb}[1][-1mu]{\permcomb[#1]{C}}

% Title page details: 
\title{Assignment 11} 
\author{Varshini Jonnala - CS21BTECH11024}
\date{\today}

\begin{document}

% Title page frame
\begin{frame}
    \titlepage 
\end{frame}

\section{Question}
\begin{frame}{Question}
Show that, if the process $X(w)$ is white noise with zero mean and autocovariance $Q(u)\delta(u - v)$, then its inverse Fourier transform $x(t)$ is WSS with power spectrum $Q(w)/2\pi$.
\end{frame}

%Blocks frame
\section{Solution}
\begin{frame}[allowframebreaks]{Solution}
    \begin{align}
        E\cbrak{x(t_1)x^*(t_2)} &=\frac{1}{4\pi^2} E\cbrak{\int_{-\infty}^\infty \int_{-\infty}^\infty  E\cbrak{X(u)X^*(v)} e^{j\brak{ut_1 - vt_2}} du dv }\\
        &= \frac{1}{4\pi^2} E\cbrak{\int_{-\infty}^\infty \int_{-\infty}^\infty  Q(u)\delta(u - v)  e^{j\brak{ut_1 - vt_2}} du dv }
       % &= \frac{1}{4\pi^2} E\cbrak{\int_{-\infty}^\infty  Q(u) e^{j\brak{ut_1 - vt_2}} du}
    \end{align}
    We know that, A random process $\cbrak{X(t),t \in R}$ whose mean is independent of $t$, is weak-sense stationary or wide-sense stationary (WSS) for all $t \in R$, if
    \begin{align}
        R_X\brak{t_1,t_2}=R_X\brak{t_1 - t_2}
    \end{align}
\newpage
    Now, 
    This depends only on $\tau = t_1 - t_2$ :
   
    
    \begin{align}
        R_{XX}\brak{\tau} &= \frac{1}{2\pi} \int_{-\infty}^\infty Q(u) e^{jur} du
    \end{align}
    Hence, Power Spectrum $S_{XX}(\omega)$ of $X(\omega)$ would be:
    \begin{align}
         S_{XX} \brak{\omega} &= \frac{Q(\omega)}{2\pi}
    \end{align}
    Hence, proved.
\end{frame} 

\end{document}
